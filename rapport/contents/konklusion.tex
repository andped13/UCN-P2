I dette projekt havde gruppen til formål at designe og implementere den nødvendige hard- og software med henblik på at konstruere en automatisk linjefølgende robot.
\\
\\
Produktet er udviklet ud fra Magician chassis som blev leveret af Sparkfun med en dertil forudbestemt lyssensor. Produktet blev udviklet over flere stadier hvoraf den indledende fase var at implementere en sensor til en Arduino og få robotten til at følge en linje ud fra den feedback som kom fra lyssensoren. 
\\
\\
Herfra har gruppen implementeret yderligere 2 lyssensorer til robotten i form af videreudvikling af produktet. Projektgruppen har foretaget test ved brug af én og 3 sensorer tilknyttet en Arduino Uno for at reducere mulige fejl. 
\\
\\
Efter succesfulde test af produktet ved brug af Arduino, har gruppen valgt at overføre projektett til MPLAB's platform. Her er der lagt særlig fokus på implementering af ADC og PWM og gjort overvejelser omkring videreudviklingen af produktet. Her er test foretaget for at verificere at ADC og PWM virker efter hensigten, desværre kan det konstateres at gruppen ikke har formået at få det til at fungere sammen. Det kan konkluderes at på grund af manglende tid har gruppen fraprioteret implementering af PID med forbehold for at et fuldt funktionelt produkt kan fremvises. \fxnote{ret}
\\
\\
Herfra har gruppen foretaget test af robotten ved brug af 3 sensorer hvor det kan konkluderes at robotten virker efter den ønskede hensigt. I følge kravspecifikationen opstillet i projektbeskrivelsen, kan det konkluderes at afstanden i mellem de 3 sensorer er for stor og robotten kan derfor ikke følge banens korteste forløb. På grund af tidsmangel er en ny 3D skabelon frapriotereret, hvilket ville løse dette problem.    

