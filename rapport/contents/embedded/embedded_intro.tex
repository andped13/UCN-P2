\section{Introduktion til embedded systems}
Med henblik på videreudvikling af produktet har gruppen valgt at lægge fokus på at overføre robottens hard- og software fra Arduino til et embedded system. Dette er gjort med forbehold for læringsmålene sat i projektbeskrivelsen, hvor ét af fokusobjekterne er på øvelse og repetition af implementering og udvikling af software i embedded systemer. 
\newline 
Medvidere her ses også mulighed for prisreducering i henhold til hardware gældende for produktet. Under den indledende process blev et Arduino Uno board anvendt for at teste den indledende problemløsning gruppen havde opstillet. 
\newline
Herfra har gruppen lagt fokus på at overføre softwaren skrevet i Arduino til MPLAP's platform. Dette har betydet at mikrocontrolleren som først blev anvendt blev udskifteet med et UCN board \ref{arduino&shield} som gruppen konstruerede under 1. semester. 
\newline
\newline
De følgende afsnit introducere hvorledes gruppen anvender og implementere ADC(analog til digital konverter) og PWM (pulsbreddemodulation) for at robotten skal virke bedst muligt i henhold til projektbeskrivelsen.  
   





