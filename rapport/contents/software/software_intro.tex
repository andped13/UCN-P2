\section{Software introduktion}
Den indlende idé til en softwareløsning som kan følge en linje er udarbejdet, og ses på figur \ref{init_software}.

I dette afsnit ses der nærmere på den software som er skrevet i takt med udviklingen af projektet. Her introduceres om andre den indledende løsningsidé samt udviklingen deraf. Herfra ses der nærmere på virdereudviklingen ad denne, fra implementering af én til flere sensorer samt betydningen deraf for produktet.
\newline 
Endvidere præsenteres test i henhold til virkningen af de forskellige stykker software som er konstrueret hertil. 

\subsection{Indledende problemløsning}
For at gøre robotten i stand til at kunne følge den opstillede bane er en indledende problemløsning opsat. Her har gruppen valgt sætte fokus på implementering af én sensor som kan videreudvikles fremadrettet. 
\newline 
Softwaren skrevet har to funktioner baseret ud fra én registreret måling. På nedenstående figur ses hvorledes softwarens handling forløber.

\begin{figure}[h!]
  \centering
  \includegraphics[width=0.6\textwidth]{figures/intisliserendeLoesning.png}
  \caption{Den inledende software løsning til linetrackinging med 1 sensor.}
  \label{init_software}
\end{figure}

Som det ses på ovenstående figur har softwaren to funktioner. Ses stregen på den opstillede bane, hvis ja drej til højre, hvis nej drej til venstre. Dette resulterer i at robotten er i stand til at kunne følge kanten af stregen ved at registrere de forskellige overfladefarvemålinger.   


