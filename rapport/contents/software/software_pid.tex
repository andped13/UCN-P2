\section{PID regulering}
For at opnå en mindskning i fejl målingerne fra sensoren,  kan indplementeres en PID. PID står for Proportional Integral Derivative controller. Formålet med en PID er at man konstant måler en fejl værdi som består af differensen mellem en ønsket værdi og en målt værdi. PID'en forsøger at minimere fejlmålingerne ved at justerer fejl målingen over tid.\newline
\newline
PID kan opskrives på følgende måde.

\begin{equation} 
u(t) = K_p e(t) + K_i\int_{0}^{t}e(T)dT + K_d\frac{\mathrm{d} e(t)}{\mathrm{d} t}
\end{equation}\label{PID}
\newline
hvor,
\newline
$K_{p}$, er proportional leddet som beskriver den nuværende værdi, hvilket er den ønskede værdi trukket fra den målte værdi. 
\newline
$K_{i}$, er integralleddet som beskriver den forhændværende værdi, hvilket betyder at fejlværdien bliver justreret ind over tid, dette gøres ved hjælp af det ønskede signal i forhold til det givet signal, som drager et areal af hver periode.   
\newline
$K_{d}$, er differentiale leddet som beskriver en mulig fremtidig fejl, som er vuderet ud fra forhændværende værdier. Dette gøres ved at måle hældningen på det givne signal for at forsøge at justere det ind efter den ønskede værdi. 
\newline

\fxnote{beslut hvad der skal gøres med afsnittet}
\newpage