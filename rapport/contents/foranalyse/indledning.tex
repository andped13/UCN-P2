\section{Indledning}
I dette semester projekt skal en robot bygges til at følge en linje. Ud fra problemformulering er noget hardware stillet til rådighed hvor gruppens formål vil være at implementere den nødvendige hard- og software, så 
robotten vil være i stand til at manøvre rundt på en oplagt linje. Produktet udvilkes af it-teknologstuderende fra University College Nordjylland på 2. Semester på elektronik linjen. 
Produkteter udvikles for at øge kompetencer og forståelser inden for allerede kendt eletorik- og programmerings viden med forbehold for at anvende det i praksis. Det færdige produkt kan ses som koncept for at behjælpe automatisering
i samfundet i form af behjælpelige maskiner til industri. 